% Options for packages loaded elsewhere
\PassOptionsToPackage{unicode}{hyperref}
\PassOptionsToPackage{hyphens}{url}
%
\documentclass[
]{article}
\usepackage{lmodern}
\usepackage{amssymb,amsmath}
\usepackage{ifxetex,ifluatex}
\ifnum 0\ifxetex 1\fi\ifluatex 1\fi=0 % if pdftex
  \usepackage[T1]{fontenc}
  \usepackage[utf8]{inputenc}
  \usepackage{textcomp} % provide euro and other symbols
\else % if luatex or xetex
  \usepackage{unicode-math}
  \defaultfontfeatures{Scale=MatchLowercase}
  \defaultfontfeatures[\rmfamily]{Ligatures=TeX,Scale=1}
\fi
% Use upquote if available, for straight quotes in verbatim environments
\IfFileExists{upquote.sty}{\usepackage{upquote}}{}
\IfFileExists{microtype.sty}{% use microtype if available
  \usepackage[]{microtype}
  \UseMicrotypeSet[protrusion]{basicmath} % disable protrusion for tt fonts
}{}
\makeatletter
\@ifundefined{KOMAClassName}{% if non-KOMA class
  \IfFileExists{parskip.sty}{%
    \usepackage{parskip}
  }{% else
    \setlength{\parindent}{0pt}
    \setlength{\parskip}{6pt plus 2pt minus 1pt}}
}{% if KOMA class
  \KOMAoptions{parskip=half}}
\makeatother
\usepackage{xcolor}
\IfFileExists{xurl.sty}{\usepackage{xurl}}{} % add URL line breaks if available
\IfFileExists{bookmark.sty}{\usepackage{bookmark}}{\usepackage{hyperref}}
\hypersetup{
  pdftitle={PrEP Optimization Summary},
  pdfauthor={Greg Knowlton, Emeli Anderson, Sam Jenness},
  hidelinks,
  pdfcreator={LaTeX via pandoc}}
\urlstyle{same} % disable monospaced font for URLs
\usepackage[margin=1in]{geometry}
\usepackage{graphicx,grffile}
\makeatletter
\def\maxwidth{\ifdim\Gin@nat@width>\linewidth\linewidth\else\Gin@nat@width\fi}
\def\maxheight{\ifdim\Gin@nat@height>\textheight\textheight\else\Gin@nat@height\fi}
\makeatother
% Scale images if necessary, so that they will not overflow the page
% margins by default, and it is still possible to overwrite the defaults
% using explicit options in \includegraphics[width, height, ...]{}
\setkeys{Gin}{width=\maxwidth,height=\maxheight,keepaspectratio}
% Set default figure placement to htbp
\makeatletter
\def\fps@figure{htbp}
\makeatother
\setlength{\emergencystretch}{3em} % prevent overfull lines
\providecommand{\tightlist}{%
  \setlength{\itemsep}{0pt}\setlength{\parskip}{0pt}}
\setcounter{secnumdepth}{-\maxdimen} % remove section numbering

\title{PrEP Optimization Summary}
\author{Greg Knowlton, Emeli Anderson, Sam Jenness}
\date{7/13/2020}

\begin{document}
\maketitle

The central question is: how should we allocate a fixed budget to three
PrEP support interventions to maximize the number of HIV infections
averted over the course of 10 years among MSM in Atlanta?

The model parameters representing policy levers being considered:

• POIP -- percentage of MSM reached by initiation intervention

• POAC\_yr -- adherence intervention capacity (max. number of people who
can receive the intervention per year)

• PORC -- retention intervention capacity (max. number of people who can
be active in the intervention in any given week)

Simulation Output Exploration:

The outcome we are looking at here is the average proportion of active
PrEP users who are highly adherent. We expect this proportion to go up
with increasing adherence intervention capacity. At a certain point,
this highly adherent proportion stops increasing with adherence capacity
because all new PrEP users are already enrolled in the adherence
intervention. When the uptake intervention is well-funded (High PrEP
Uptake), a greater adherence capacity can be utilized. Consequently, we
see the proportion of active PrEP users who are highly adherent start to
plateau at a lower adherence capacity when PrEP uptake is also low.
``Low PrEP Uptake'' and ``High PrEP Uptake'' are coded as the
simulations with the lowest and highest quartiles of the parameter that
controls the initiation intervention capacity.

\includegraphics{Optim_Summary_files/figure-latex/high_adherence_plot-1.pdf}

The y-axis denotes the mean proportion of the PrEP eligible population
that is actively taking PrEP for each simulation over the course of the
10 year time horizon. Increasing the rate of PrEP uptake and the average
length of time each PrEP user stays on the medication will both raise
overall coverage levels. Increasing retention intervention capacity
raises coverage levels until the capacity exceeds the number of PrEP
users who can utilize the capacity. Increasing PrEP uptake shifts this
point of excessive retention capacity to a higher value. Visually
speaking, it appears that marginal effect of retention capacity on PrEP
coverage is the same at all levels of PrEP uptake, except in the space
where retention capacity exceeds demand. If infections averted is a
function of PrEP coverage, then this suggests that the lack of a
synergistic interaction between the uptake and retention interventions
when retention demand is not exceeded. ``Low PrEP Uptake'' and ``High
PrEP Uptake'' are coded as the simulations with the lowest and highest
quartiles of the parameter that controls the initiation intervention
capacity.

\includegraphics{Optim_Summary_files/figure-latex/pCov_porc_plot-1.pdf}
This figure shows the same dynamics in a slightly different way. The
parameter represented by the x-axis (PrEP Intervention Initiation
Percentage) is the percentage of individuals newly indicated for PrEP
who become actively engaged in the initiation intervention. The higher
retention capacity values don't lead to an increase in PrEP coverage,
and the relationship between coverage levels and the initiation
intervention capacity parameter appears to be roughly linear and
independent of retention capacity.

\includegraphics{Optim_Summary_files/figure-latex/pCov_poip_plot-1.pdf}
So far, we have only described the relationship between the parameters
controlling the capacity of PrEP support interventions and intermediate
outcomes in an exploratory fashion. To answer our research question
within a mathematical optimization framework, we need to reduce the the
messiness of the simulation data down to smooth functions of the policy
parameters of interest.

We use a generalized additive model (GAM) to fit a flexible and smooth
function to our simulation output. In this statistical modeling
approach, our outcome is cumulative infections averted, and the
predictors of interest are uptake program initiation proportion (POIP),
retention program capacity (PORC), and adherence program capacity
(POAC\_yr). Like generalized linear models, GAMs are specified by a link
function and a mean variance relationship (family). Because our outcome
(infections averted) is positive and continuous, we used a Gamma family
mean-variance relationship, and we chose to use a log link function
because it provided the best fit to our data in terms of GCV score. In
GAMs, relationships between the individual predictors and the dependent
variable follow smooth patterns that can be nonlinear.

We can visualize the generalized additive model predictions for
infections averted to get a sense of how the program parameters interact
and affect outcomes. On these plots, ``response'' is infections averted,
and on the three dimensional plots, the axes span the entire allowable
range of each capacity parameter (POIP: 0-0.10, POAC\_yr: 0-2000, PORC:
0-2000).

As a reminder:

• POIP -- percentage of MSM reached by initiation intervention

• POAC\_yr -- adherence intervention capacity (max. number of people who
can receive the intervention per year)

• PORC -- retention intervention capacity (max. number of people who can
be active in the intervention in any given week)

The ``response'' on these plots (the z-axis height and the value on the
contour lines) is the predicted number of infections averted.

Note: These interaction plots are made automatically through the mgcv
package, and I will work on customizing these plots to improve their
readability, translate the axes into economic terms, and show the path
of the optimal resource allocation as a parametric function of budget
size.

Uptake/Retention Interaction plots:

\includegraphics{Optim_Summary_files/figure-latex/vis_gam-1.pdf}
\includegraphics{Optim_Summary_files/figure-latex/vis_gam-2.pdf}

Uptake/Adherence Interaction plots:

\includegraphics{Optim_Summary_files/figure-latex/unnamed-chunk-1-1.pdf}
\includegraphics{Optim_Summary_files/figure-latex/unnamed-chunk-1-2.pdf}

Adherence/Retention Interaction plots:

\includegraphics{Optim_Summary_files/figure-latex/unnamed-chunk-2-1.pdf}
\includegraphics{Optim_Summary_files/figure-latex/unnamed-chunk-2-2.pdf}

The y-axis denotes the predicted number of infections averted based on
the generalized additive model and three different intervention capacity
values. The plot has three facets for the minimum, median, and maximum
values for the uptake intervention parameter, and each curve is
color-coded according the minimum, median, and maximum adherence
intervention capacity. This allows one to visualize the effect of
changing retention capacity on infections averted at various
combinations of the other program capacities.

\includegraphics{Optim_Summary_files/figure-latex/unnamed-chunk-3-1.pdf}

These plots are faceted by uptake intervention capacity and color coded
according to retention intervention capacity. We are able to see how
adjusting retention intervention capacity affects predicted infections
averted at representative combinations of the other two intervention
capacity parameters.

\includegraphics{Optim_Summary_files/figure-latex/unnamed-chunk-4-1.pdf}

These plots are faceted by adherence intervention capacity, and color
coded by retention intervention capacity. We are able to see how
adjusting uptake intervention capacity affects predicted infections
averted at representative combinations of the other two intervention
capacity parameters.

\includegraphics{Optim_Summary_files/figure-latex/unnamed-chunk-5-1.pdf}

The generalized additive model informs the leverage of each intervention
capacity parameter in maximizing infections averted, but we also need to
define the monetary costs of increasing the capacities for each
intervention in order to translate the optimization problem into
economic terms. The following linear unit costs are assumed for the
budget optimization problem:

\textasciitilde\$632,000 per percentage point increase to PrEP
Optimization Initiation Proportion \textasciitilde\$5,100 per yearly
adherence intervention capacity slot \textasciitilde\$720 per retention
intervention capacity slot (each slot of capacity used for an average of
337 days)

These costs are based in micro-costing data and the resources described
in the clinical trial protocols. There is uncertainty for all unit
costs, but this uncertainty is most significant for the
uptake/initiation intervention parameter that controls the proportion of
newly indicated PrEP users who become actively engaged in the
intervention. The specific unit costs assumed in the following budget
optimization figures are within the range of feasible values, but are
also chosen to illustrate some of the important interactions between the
interventions.

With the assumed unit costs and the model-derived predictions for
infections averted as a function of the intervention capacities, we use
a nonlinear optimization algorithm to solve for the optimal allocation
of funds at a range of budgets from \$700,000 to \$6,000,000.

In this first plot of budget optimization results, we are looking only
at how our predicted number of infections averted increases as a
function of the budget constraint. Without seeing how the budget is
allocated, this plot tells us how much overall predicted value is
extracted from a given budget.

\includegraphics{Optim_Summary_files/figure-latex/unnamed-chunk-6-1.pdf}

The next three plots will show the optimal capacity parameter values at
a range of budget constraints. Note that the range of allowable
adherence capacity is 0-2000, and the noise around an adherence capacity
of 0 means that the adherence intervention never represented a
cost-efficient use of funds at any budget value. One next step will be
to find the cutoff point in the unit cost of the adherence intervention
at which the intervention becomes viable to some extent in the budget
range being considered, because it clearly doesn't not enter into the
realm of possibilities under the current assumptions.

\includegraphics{Optim_Summary_files/figure-latex/unnamed-chunk-7-1.pdf}

At low budget values, it is not optimal to allocate funding to the
uptake intervention. Once the retention intervention starts to give
significantly diminished returns, injecting money into the uptake
intervention becomes optimal. Once the uptake/initiation intervention
starts to become funded around a \$1M budget, the level of funding
appears to increase linearly as the budget increases, but that is not
exactly the case.

\includegraphics{Optim_Summary_files/figure-latex/unnamed-chunk-8-1.pdf}

At low budgets, the retention intervention is the most cost-efficient
and absorbs the entire budget. Around a retention capacity of 1,400, the
population of active PrEP users has been effectively saturated in its
engagement with the retention intervention. The only way to extract
significant additional value from the retention intervention is to also
increase the number of individuals who are on PrEP, and the uptake
intervention provides a means for achieving this. As the uptake
intervention raises the size of the PrEP active population, it opens the
door for additional cost-efficient funding of the retention
intervention.

\includegraphics{Optim_Summary_files/figure-latex/unnamed-chunk-9-1.pdf}

The following plot shows the fraction of the budget that is allocated to
each intervention capacity as a function of the total budget size:

\includegraphics{Optim_Summary_files/figure-latex/unnamed-chunk-10-1.pdf}

This plot shows the same general relationship, except the y axis is the
total amount of funding rather than the fraction of funding allocated to
each intervention capacity across various budget sizes. Note that the
funding for the retention intervention rises slightly at higher budgets
as the initiation intervention raises the ceiling for usable retention
capacity.

\includegraphics{Optim_Summary_files/figure-latex/unnamed-chunk-11-1.pdf}

\end{document}
